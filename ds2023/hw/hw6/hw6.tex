% Options for packages loaded elsewhere
% Options for packages loaded elsewhere
\PassOptionsToPackage{unicode}{hyperref}
\PassOptionsToPackage{hyphens}{url}
\PassOptionsToPackage{dvipsnames,svgnames,x11names}{xcolor}
%
\documentclass[
  letterpaper,
  DIV=11,
  numbers=noendperiod]{scrartcl}
\usepackage{xcolor}
\usepackage{amsmath,amssymb}
\setcounter{secnumdepth}{-\maxdimen} % remove section numbering
\usepackage{iftex}
\ifPDFTeX
  \usepackage[T1]{fontenc}
  \usepackage[utf8]{inputenc}
  \usepackage{textcomp} % provide euro and other symbols
\else % if luatex or xetex
  \usepackage{unicode-math} % this also loads fontspec
  \defaultfontfeatures{Scale=MatchLowercase}
  \defaultfontfeatures[\rmfamily]{Ligatures=TeX,Scale=1}
\fi
\usepackage{lmodern}
\ifPDFTeX\else
  % xetex/luatex font selection
\fi
% Use upquote if available, for straight quotes in verbatim environments
\IfFileExists{upquote.sty}{\usepackage{upquote}}{}
\IfFileExists{microtype.sty}{% use microtype if available
  \usepackage[]{microtype}
  \UseMicrotypeSet[protrusion]{basicmath} % disable protrusion for tt fonts
}{}
\makeatletter
\@ifundefined{KOMAClassName}{% if non-KOMA class
  \IfFileExists{parskip.sty}{%
    \usepackage{parskip}
  }{% else
    \setlength{\parindent}{0pt}
    \setlength{\parskip}{6pt plus 2pt minus 1pt}}
}{% if KOMA class
  \KOMAoptions{parskip=half}}
\makeatother
% Make \paragraph and \subparagraph free-standing
\makeatletter
\ifx\paragraph\undefined\else
  \let\oldparagraph\paragraph
  \renewcommand{\paragraph}{
    \@ifstar
      \xxxParagraphStar
      \xxxParagraphNoStar
  }
  \newcommand{\xxxParagraphStar}[1]{\oldparagraph*{#1}\mbox{}}
  \newcommand{\xxxParagraphNoStar}[1]{\oldparagraph{#1}\mbox{}}
\fi
\ifx\subparagraph\undefined\else
  \let\oldsubparagraph\subparagraph
  \renewcommand{\subparagraph}{
    \@ifstar
      \xxxSubParagraphStar
      \xxxSubParagraphNoStar
  }
  \newcommand{\xxxSubParagraphStar}[1]{\oldsubparagraph*{#1}\mbox{}}
  \newcommand{\xxxSubParagraphNoStar}[1]{\oldsubparagraph{#1}\mbox{}}
\fi
\makeatother

\usepackage{color}
\usepackage{fancyvrb}
\newcommand{\VerbBar}{|}
\newcommand{\VERB}{\Verb[commandchars=\\\{\}]}
\DefineVerbatimEnvironment{Highlighting}{Verbatim}{commandchars=\\\{\}}
% Add ',fontsize=\small' for more characters per line
\usepackage{framed}
\definecolor{shadecolor}{RGB}{241,243,245}
\newenvironment{Shaded}{\begin{snugshade}}{\end{snugshade}}
\newcommand{\AlertTok}[1]{\textcolor[rgb]{0.68,0.00,0.00}{#1}}
\newcommand{\AnnotationTok}[1]{\textcolor[rgb]{0.37,0.37,0.37}{#1}}
\newcommand{\AttributeTok}[1]{\textcolor[rgb]{0.40,0.45,0.13}{#1}}
\newcommand{\BaseNTok}[1]{\textcolor[rgb]{0.68,0.00,0.00}{#1}}
\newcommand{\BuiltInTok}[1]{\textcolor[rgb]{0.00,0.23,0.31}{#1}}
\newcommand{\CharTok}[1]{\textcolor[rgb]{0.13,0.47,0.30}{#1}}
\newcommand{\CommentTok}[1]{\textcolor[rgb]{0.37,0.37,0.37}{#1}}
\newcommand{\CommentVarTok}[1]{\textcolor[rgb]{0.37,0.37,0.37}{\textit{#1}}}
\newcommand{\ConstantTok}[1]{\textcolor[rgb]{0.56,0.35,0.01}{#1}}
\newcommand{\ControlFlowTok}[1]{\textcolor[rgb]{0.00,0.23,0.31}{\textbf{#1}}}
\newcommand{\DataTypeTok}[1]{\textcolor[rgb]{0.68,0.00,0.00}{#1}}
\newcommand{\DecValTok}[1]{\textcolor[rgb]{0.68,0.00,0.00}{#1}}
\newcommand{\DocumentationTok}[1]{\textcolor[rgb]{0.37,0.37,0.37}{\textit{#1}}}
\newcommand{\ErrorTok}[1]{\textcolor[rgb]{0.68,0.00,0.00}{#1}}
\newcommand{\ExtensionTok}[1]{\textcolor[rgb]{0.00,0.23,0.31}{#1}}
\newcommand{\FloatTok}[1]{\textcolor[rgb]{0.68,0.00,0.00}{#1}}
\newcommand{\FunctionTok}[1]{\textcolor[rgb]{0.28,0.35,0.67}{#1}}
\newcommand{\ImportTok}[1]{\textcolor[rgb]{0.00,0.46,0.62}{#1}}
\newcommand{\InformationTok}[1]{\textcolor[rgb]{0.37,0.37,0.37}{#1}}
\newcommand{\KeywordTok}[1]{\textcolor[rgb]{0.00,0.23,0.31}{\textbf{#1}}}
\newcommand{\NormalTok}[1]{\textcolor[rgb]{0.00,0.23,0.31}{#1}}
\newcommand{\OperatorTok}[1]{\textcolor[rgb]{0.37,0.37,0.37}{#1}}
\newcommand{\OtherTok}[1]{\textcolor[rgb]{0.00,0.23,0.31}{#1}}
\newcommand{\PreprocessorTok}[1]{\textcolor[rgb]{0.68,0.00,0.00}{#1}}
\newcommand{\RegionMarkerTok}[1]{\textcolor[rgb]{0.00,0.23,0.31}{#1}}
\newcommand{\SpecialCharTok}[1]{\textcolor[rgb]{0.37,0.37,0.37}{#1}}
\newcommand{\SpecialStringTok}[1]{\textcolor[rgb]{0.13,0.47,0.30}{#1}}
\newcommand{\StringTok}[1]{\textcolor[rgb]{0.13,0.47,0.30}{#1}}
\newcommand{\VariableTok}[1]{\textcolor[rgb]{0.07,0.07,0.07}{#1}}
\newcommand{\VerbatimStringTok}[1]{\textcolor[rgb]{0.13,0.47,0.30}{#1}}
\newcommand{\WarningTok}[1]{\textcolor[rgb]{0.37,0.37,0.37}{\textit{#1}}}

\usepackage{longtable,booktabs,array}
\usepackage{calc} % for calculating minipage widths
% Correct order of tables after \paragraph or \subparagraph
\usepackage{etoolbox}
\makeatletter
\patchcmd\longtable{\par}{\if@noskipsec\mbox{}\fi\par}{}{}
\makeatother
% Allow footnotes in longtable head/foot
\IfFileExists{footnotehyper.sty}{\usepackage{footnotehyper}}{\usepackage{footnote}}
\makesavenoteenv{longtable}
\usepackage{graphicx}
\makeatletter
\newsavebox\pandoc@box
\newcommand*\pandocbounded[1]{% scales image to fit in text height/width
  \sbox\pandoc@box{#1}%
  \Gscale@div\@tempa{\textheight}{\dimexpr\ht\pandoc@box+\dp\pandoc@box\relax}%
  \Gscale@div\@tempb{\linewidth}{\wd\pandoc@box}%
  \ifdim\@tempb\p@<\@tempa\p@\let\@tempa\@tempb\fi% select the smaller of both
  \ifdim\@tempa\p@<\p@\scalebox{\@tempa}{\usebox\pandoc@box}%
  \else\usebox{\pandoc@box}%
  \fi%
}
% Set default figure placement to htbp
\def\fps@figure{htbp}
\makeatother





\setlength{\emergencystretch}{3em} % prevent overfull lines

\providecommand{\tightlist}{%
  \setlength{\itemsep}{0pt}\setlength{\parskip}{0pt}}



 


\KOMAoption{captions}{tableheading}
\makeatletter
\@ifpackageloaded{caption}{}{\usepackage{caption}}
\AtBeginDocument{%
\ifdefined\contentsname
  \renewcommand*\contentsname{Table of contents}
\else
  \newcommand\contentsname{Table of contents}
\fi
\ifdefined\listfigurename
  \renewcommand*\listfigurename{List of Figures}
\else
  \newcommand\listfigurename{List of Figures}
\fi
\ifdefined\listtablename
  \renewcommand*\listtablename{List of Tables}
\else
  \newcommand\listtablename{List of Tables}
\fi
\ifdefined\figurename
  \renewcommand*\figurename{Figure}
\else
  \newcommand\figurename{Figure}
\fi
\ifdefined\tablename
  \renewcommand*\tablename{Table}
\else
  \newcommand\tablename{Table}
\fi
}
\@ifpackageloaded{float}{}{\usepackage{float}}
\floatstyle{ruled}
\@ifundefined{c@chapter}{\newfloat{codelisting}{h}{lop}}{\newfloat{codelisting}{h}{lop}[chapter]}
\floatname{codelisting}{Listing}
\newcommand*\listoflistings{\listof{codelisting}{List of Listings}}
\makeatother
\makeatletter
\makeatother
\makeatletter
\@ifpackageloaded{caption}{}{\usepackage{caption}}
\@ifpackageloaded{subcaption}{}{\usepackage{subcaption}}
\makeatother
\usepackage{bookmark}
\IfFileExists{xurl.sty}{\usepackage{xurl}}{} % add URL line breaks if available
\urlstyle{same}
\hypersetup{
  pdftitle={DS 2023 HW 6},
  colorlinks=true,
  linkcolor={blue},
  filecolor={Maroon},
  citecolor={Blue},
  urlcolor={Blue},
  pdfcreator={LaTeX via pandoc}}


\title{DS 2023 HW 6}
\author{}
\date{}
\begin{document}
\maketitle


\subsection{Instructions}\label{instructions}

Follow the prompts in this notebook.

Make use of code provided.

Respond with code to generate your answers. \textbar{} If asked to
provide a specific response, create a Markdown cell after your code
cell(s) with the response.

NOTE: To save as a PDF, first export your notebook to HTML, open it in a
brower, and the print the web page to PDF.

\textbf{10 Points}

\subsection{Student Info}\label{student-info}

Write your name and computer ID below.

\begin{itemize}
\tightlist
\item
  NAME: Sae-Jin Moon
\item
  ID: rhn9qs
\end{itemize}

\subsection{Compliance with Homework Submission
Guidelines}\label{compliance-with-homework-submission-guidelines}

1 pt

\subsection{Set Up}\label{set-up}

\begin{Shaded}
\begin{Highlighting}[]
\ImportTok{import}\NormalTok{ pandas }\ImportTok{as}\NormalTok{ pd}
\ImportTok{import}\NormalTok{ numpy }\ImportTok{as}\NormalTok{ np}
\ImportTok{import}\NormalTok{ matplotlib.pyplot }\ImportTok{as}\NormalTok{ plt}
\ImportTok{import}\NormalTok{ seaborn }\ImportTok{as}\NormalTok{ sns}
\end{Highlighting}
\end{Shaded}

\begin{Shaded}
\begin{Highlighting}[]
\NormalTok{sns.set\_theme(style}\OperatorTok{=}\StringTok{\textquotesingle{}darkgrid\textquotesingle{}}\NormalTok{, context}\OperatorTok{=}\StringTok{\textquotesingle{}notebook\textquotesingle{}}\NormalTok{) }
\end{Highlighting}
\end{Shaded}

Read in the data file as a data frame called \texttt{DOC}.

\begin{Shaded}
\begin{Highlighting}[]
\NormalTok{data\_src }\OperatorTok{=} \StringTok{"winereviews{-}DOC.csv"}
\NormalTok{DOC }\OperatorTok{=}\NormalTok{ pd.read\_csv(data\_src, index\_col}\OperatorTok{=}\StringTok{\textquotesingle{}doc\_id\textquotesingle{}}\NormalTok{)}
\end{Highlighting}
\end{Shaded}

\subsection{Tasks}\label{tasks}

\subsubsection{Task 1}\label{task-1}

1 pt

Create a data frame called \texttt{COUNTRY} that contains a row for each
distinct value in \texttt{doc\_country} and a column for the number of
reviews (rows) associated with each country.

Display the first \(5\) rows.

\begin{Shaded}
\begin{Highlighting}[]
\NormalTok{COUNTRY }\OperatorTok{=}\NormalTok{ DOC[}\StringTok{\textquotesingle{}doc\_country\textquotesingle{}}\NormalTok{].value\_counts().to\_frame(}\StringTok{\textquotesingle{}count\textquotesingle{}}\NormalTok{)}
\NormalTok{COUNTRY.head()}
\end{Highlighting}
\end{Shaded}

\begin{longtable}[]{@{}ll@{}}
\toprule\noalign{}
& count \\
doc\_country & \\
\midrule\noalign{}
\endhead
\bottomrule\noalign{}
\endlastfoot
US & 54504 \\
France & 22093 \\
Italy & 19540 \\
Spain & 6645 \\
Portugal & 5691 \\
\end{longtable}

\subsubsection{Task 2}\label{task-2}

1 pt

Create a \textbf{count} plot showing the review count for each country,
were the country names on the x-axis are \textbf{sorted} from highest to
lowest counts and rotated \(90\degree\).

\begin{Shaded}
\begin{Highlighting}[]
\NormalTok{plt.figure(figsize}\OperatorTok{=}\NormalTok{(}\DecValTok{12}\NormalTok{, }\DecValTok{6}\NormalTok{))}
\NormalTok{sns.countplot(x}\OperatorTok{=}\StringTok{\textquotesingle{}doc\_country\textquotesingle{}}\NormalTok{, data}\OperatorTok{=}\NormalTok{DOC, order}\OperatorTok{=}\NormalTok{COUNTRY.index)}
\NormalTok{plt.xticks(rotation}\OperatorTok{=}\DecValTok{90}\NormalTok{)}
\NormalTok{plt.title(}\StringTok{\textquotesingle{}Review Count by Country\textquotesingle{}}\NormalTok{)}
\NormalTok{plt.xlabel(}\StringTok{\textquotesingle{}Country\textquotesingle{}}\NormalTok{)}
\NormalTok{plt.ylabel(}\StringTok{\textquotesingle{}Number of Reviews\textquotesingle{}}\NormalTok{)}
\NormalTok{plt.show()}
\end{Highlighting}
\end{Shaded}

\pandocbounded{\includegraphics[keepaspectratio]{hw6_files/figure-pdf/cell-6-output-1.png}}

\subsubsection{Task 3}\label{task-3}

1 pt

By eye, what are the three most represented countries in the plot above,
in desceding order of count?

Write your answer as a comma separated list below.

US, France, Italy

\subsubsection{Task 4}\label{task-4}

1 pt

Create a box plot of the counts in the \texttt{COUNTRY} data frame.

\begin{Shaded}
\begin{Highlighting}[]
\NormalTok{plt.figure(figsize}\OperatorTok{=}\NormalTok{(}\DecValTok{10}\NormalTok{, }\DecValTok{2}\NormalTok{))}
\NormalTok{sns.boxplot(x}\OperatorTok{=}\NormalTok{COUNTRY[}\StringTok{"count"}\NormalTok{])}
\NormalTok{plt.title(}\StringTok{\textquotesingle{}Distribution of Review Counts Across Countries\textquotesingle{}}\NormalTok{)}
\NormalTok{plt.xlabel(}\StringTok{\textquotesingle{}Number of Reviews\textquotesingle{}}\NormalTok{)}
\NormalTok{plt.show()}
\end{Highlighting}
\end{Shaded}

\pandocbounded{\includegraphics[keepaspectratio]{hw6_files/figure-pdf/cell-7-output-1.png}}

\subsubsection{Task 5}\label{task-5}

1 pt

Add two features to \texttt{COUNTRY}, one for the mean value of
\texttt{doc\_points} and one for the \textbf{median} value of
\texttt{doc\_price}.

\begin{Shaded}
\begin{Highlighting}[]
\NormalTok{country\_stats }\OperatorTok{=}\NormalTok{ DOC.groupby(}\StringTok{\textquotesingle{}doc\_country\textquotesingle{}}\NormalTok{).agg(}
\NormalTok{    mean\_points}\OperatorTok{=}\NormalTok{(}\StringTok{\textquotesingle{}doc\_points\textquotesingle{}}\NormalTok{, }\StringTok{\textquotesingle{}mean\textquotesingle{}}\NormalTok{),}
\NormalTok{    median\_price}\OperatorTok{=}\NormalTok{(}\StringTok{\textquotesingle{}doc\_price\textquotesingle{}}\NormalTok{, }\StringTok{\textquotesingle{}median\textquotesingle{}}\NormalTok{)}
\NormalTok{)}
\NormalTok{COUNTRY }\OperatorTok{=}\NormalTok{ COUNTRY.join(country\_stats)}
\NormalTok{COUNTRY.head()}
\end{Highlighting}
\end{Shaded}

\begin{longtable}[]{@{}llll@{}}
\toprule\noalign{}
& count & mean\_points & median\_price \\
doc\_country & & & \\
\midrule\noalign{}
\endhead
\bottomrule\noalign{}
\endlastfoot
US & 54504 & 88.563720 & 30.0 \\
France & 22093 & 88.845109 & 25.0 \\
Italy & 19540 & 88.562231 & 28.0 \\
Spain & 6645 & 87.288337 & 18.0 \\
Portugal & 5691 & 88.250220 & 16.0 \\
\end{longtable}

\subsubsection{Task 6}\label{task-6}

1 pt

Using the table you just created, create a scatter plot that compares
median \texttt{doc\_price} to mean \texttt{doc\_points}.

Use the pattern found in Wednesday's activity key to put country labels
on each point.

\begin{Shaded}
\begin{Highlighting}[]
\NormalTok{plt.figure(figsize}\OperatorTok{=}\NormalTok{(}\DecValTok{12}\NormalTok{, }\DecValTok{8}\NormalTok{))}
\NormalTok{sns.scatterplot(data}\OperatorTok{=}\NormalTok{COUNTRY, x}\OperatorTok{=}\StringTok{\textquotesingle{}mean\_points\textquotesingle{}}\NormalTok{, y}\OperatorTok{=}\StringTok{\textquotesingle{}median\_price\textquotesingle{}}\NormalTok{)}

\ControlFlowTok{for}\NormalTok{ i, row }\KeywordTok{in}\NormalTok{ COUNTRY.iterrows():}
\NormalTok{    plt.annotate(i, (row[}\StringTok{\textquotesingle{}mean\_points\textquotesingle{}}\NormalTok{], row[}\StringTok{\textquotesingle{}median\_price\textquotesingle{}}\NormalTok{]), xytext}\OperatorTok{=}\NormalTok{(}\DecValTok{3}\NormalTok{, }\DecValTok{3}\NormalTok{), textcoords}\OperatorTok{=}\StringTok{\textquotesingle{}offset points\textquotesingle{}}\NormalTok{)}


\NormalTok{plt.title(}\StringTok{\textquotesingle{}Median Price vs. Mean Points by Country\textquotesingle{}}\NormalTok{)}
\NormalTok{plt.xlabel(}\StringTok{\textquotesingle{}Mean Points\textquotesingle{}}\NormalTok{)}
\NormalTok{plt.ylabel(}\StringTok{\textquotesingle{}Median Price\textquotesingle{}}\NormalTok{)}
\NormalTok{plt.show()}
\end{Highlighting}
\end{Shaded}

\pandocbounded{\includegraphics[keepaspectratio]{hw6_files/figure-pdf/cell-9-output-1.png}}

\subsubsection{Task 7}\label{task-7}

1 pt

Which country stands out as having the highest x and y values?

England

\subsubsection{Task 8}\label{task-8}

1 pt

Create a wide data frame called \texttt{TASTER\_COUNTRY} with the
following properties:

\begin{itemize}
\tightlist
\item
  axis \(0\) contains the domain of \texttt{doc\_taster}
\item
  axis \(1\) contains the domain of \texttt{doc\_country}
\item
  cells contain counts for the co-occurrence of \texttt{doc\_taster} and
  \texttt{doc\_country}
\end{itemize}

\begin{Shaded}
\begin{Highlighting}[]
\NormalTok{TASTER\_COUNTRY }\OperatorTok{=}\NormalTok{ pd.crosstab(DOC[}\StringTok{\textquotesingle{}doc\_taster\textquotesingle{}}\NormalTok{], DOC[}\StringTok{\textquotesingle{}doc\_country\textquotesingle{}}\NormalTok{])}
\NormalTok{TASTER\_COUNTRY.head()}
\end{Highlighting}
\end{Shaded}

\begin{longtable}[]{@{}llllllllllllllllllllll@{}}
\toprule\noalign{}
doc\_country & Argentina & Armenia & Australia & Austria & Bosnia and
Herzegovina & Brazil & Bulgaria & Canada & Chile & China & ... & Serbia
& Slovakia & Slovenia & South Africa & Spain & Switzerland & Turkey & US
& Ukraine & Uruguay \\
doc\_taster & & & & & & & & & & & & & & & & & & & & & \\
\midrule\noalign{}
\endhead
\bottomrule\noalign{}
\endlastfoot
Alexander Peartree & 0 & 0 & 0 & 0 & 0 & 0 & 0 & 0 & 0 & 0 & ... & 0 & 0
& 0 & 0 & 0 & 0 & 0 & 415 & 0 & 0 \\
Anna Lee C. Iijima & 0 & 0 & 0 & 0 & 1 & 0 & 19 & 15 & 0 & 0 & ... & 0 &
1 & 26 & 0 & 0 & 0 & 5 & 2435 & 2 & 0 \\
Anne Krebiehl~MW & 0 & 0 & 0 & 2207 & 0 & 0 & 0 & 0 & 0 & 0 & ... & 0 &
0 & 0 & 0 & 0 & 0 & 0 & 0 & 0 & 0 \\
Carrie Dykes & 0 & 0 & 0 & 0 & 0 & 0 & 0 & 0 & 0 & 0 & ... & 0 & 0 & 0 &
0 & 0 & 0 & 0 & 139 & 0 & 0 \\
Christina Pickard & 0 & 0 & 1 & 0 & 0 & 0 & 0 & 0 & 0 & 0 & ... & 0 & 0
& 0 & 0 & 0 & 0 & 0 & 5 & 0 & 0 \\
\end{longtable}

\subsubsection{Task 9}\label{task-9}

1 pt

Create a data frame \texttt{LATIN} by executing the code below.

\begin{Shaded}
\begin{Highlighting}[]
\NormalTok{latin\_countries }\OperatorTok{=} \StringTok{"Uruguay Argentina Chile Brazil Peru Mexico Spain"}\NormalTok{.split()}
\NormalTok{LATIN }\OperatorTok{=}\NormalTok{ TASTER\_COUNTRY[latin\_countries].stack().to\_frame(}\StringTok{\textquotesingle{}count\textquotesingle{}}\NormalTok{)}
\end{Highlighting}
\end{Shaded}

Create a bar plot with \texttt{LATIN} that compares \texttt{doc\_taster}
and \texttt{count}.

\begin{Shaded}
\begin{Highlighting}[]
\CommentTok{\# ADD CODE HERE}
\NormalTok{LATIN\_for\_plot }\OperatorTok{=}\NormalTok{ LATIN.loc[LATIN[}\StringTok{\textquotesingle{}count\textquotesingle{}}\NormalTok{] }\OperatorTok{\textgreater{}} \DecValTok{0}\NormalTok{].reset\_index()}

\NormalTok{plt.figure(figsize}\OperatorTok{=}\NormalTok{(}\DecValTok{12}\NormalTok{, }\DecValTok{7}\NormalTok{))}
\NormalTok{sns.barplot(data}\OperatorTok{=}\NormalTok{LATIN\_for\_plot, x}\OperatorTok{=}\StringTok{\textquotesingle{}doc\_taster\textquotesingle{}}\NormalTok{, y}\OperatorTok{=}\StringTok{\textquotesingle{}count\textquotesingle{}}\NormalTok{, hue}\OperatorTok{=}\StringTok{\textquotesingle{}doc\_country\textquotesingle{}}\NormalTok{)}
\NormalTok{plt.xticks(rotation}\OperatorTok{=}\DecValTok{90}\NormalTok{)}
\NormalTok{plt.title(}\StringTok{\textquotesingle{}Review Counts for Latin Countries by Taster\textquotesingle{}}\NormalTok{)}
\NormalTok{plt.xlabel(}\StringTok{\textquotesingle{}Taster\textquotesingle{}}\NormalTok{)}
\NormalTok{plt.ylabel(}\StringTok{\textquotesingle{}Number of Reviews\textquotesingle{}}\NormalTok{)}
\NormalTok{plt.tight\_layout()}
\NormalTok{plt.show()}
\end{Highlighting}
\end{Shaded}

\pandocbounded{\includegraphics[keepaspectratio]{hw6_files/figure-pdf/cell-12-output-1.png}}

How many tasters appear to be significantly involved with writing
reviews about the countries in \texttt{LATIN}?

Michael Schachner is the most significantly involved with writing
reviews about the countries in \texttt{LATIN}.




\end{document}
